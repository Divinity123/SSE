Searchable encryption aims to store user data remotely and facilitate searching functionality on the data, while protecting user privacy. Searchable encryption schemes are categorised into symmetric searchable encryption (SSE) schemes and public key encryption schemes with keyword search (PEKS). For SSE schemes, only the user with private key can amend the encrypted data structure and issue encrypted queries, whereas PEKS allows any user with public-key to insert new data into the encrypted data structure but only the users with the private keys can search. Clearly PEKS is more powerful than SSE, but due to the public-key nature, schemes in this category often require more complicated techniques and are less efficient in practice. For our interest, we focus on SSE schemes.


Over the past few years, a multitude of SSE schemes have been proposed, with different trade-offs between efficiency and security, and rich functionalities beyond single keyword search. However, many security vulnerabilities have been found on the schemes. Some of the attacks proposed in the literature requires very little prior knowledge and computing power, and are able to decipher overwhelming amount of encrypted data. All these attacks exploits what is called leakage of the schemes, which are permitted by the security notion. This leads us to ask:

\begin{itemize}
	\item Why does the attacks work in the presence of current security notions?
	\item Is there a security notion that captures different classes of attacks?
\end{itemize}

We argue that the current notions are insufficient to capture known attacks. We device new notion to tackle the problem. We propose new schemes and analyse their security with respect to our new notion.
\subsection{Discussion}
We will now address the question why the adversary can still mount those attacks even though the schemes in concern are proven to be secure under SS-CKA. The reason is quite simple: the notion captures the idea that whatever happens in the real world can be simulated by the ideal process. Assume that there is an attack in the real world, then SS-CKA guarantees that there is an attack that performs equally well in the ideal world, and it does not quantify if an attack is viable in the first place. In the worst case, consider an adversary who maintains his state as the database he challenged, then he is able to do everything he wants in the real world, but he does it equally well in the ideal world, and SS-CKA is not violated.

From the discussion above, we identify two shortcomings of SS-CKA. First of all, SS-CKA only parametrise over the leakage function and does not treat adversaries with different prior knowledges on the database differently. Secondly, it is not enough to prove that the encryption scheme can be simulated by some ideal process. We need to quantify precisely how well the adversary does under his prior knowledge.

In the next section, we will begin by defining syntax of SSE, and then give our definition of security. In the sections after, we will investigate basic properties of our security notion, and show how our notion can be used.
Searchable encryption aims to store user data remotely and facilitate searching functionality on the data, while protecting user privacy. Searchable encryption schemes are categorised into symmetric searchable encryption (SSE) schemes and public key encryption schemes with keyword search (PEKS). For SSE schemes, only the user with private key can amend the encrypted data structure and issue encrypted queries, whereas PEKS allows any user with public-key to insert new data into the encrypted data structure but only the users with the private keys can search. Clearly PEKS is more powerful than SSE, but due to the public-key nature, schemes in this category often require more complicated techniques and are less efficient in practice. For our interest, we focus on SSE schemes.


Over the past few years, a multitude of SSE schemes have been proposed, with different trade-offs between efficiency and security, and rich functionalities beyond single keyword search. However, many security vulnerabilities have been found on the schemes. Some of the attacks proposed in the literature requires very little prior knowledge and computing power, and are able to decipher overwhelming amount of encrypted data. All these attacks exploits what is called leakage of the schemes, which are permitted by the security notion.


In this report, we begin by introducing the readers to security notions for SSE and outline a few constructions satisfying the notions. Then we list existing attacks on SSE schemes. Most of the attacks are leakage-based attacks, meaning that the attacks will work against a class of schemes leaking at least the required amount of information. After that, we will explain why leakage abuse attacks are not captured by current notions.

Our work attempts to address the problems of current notions by proposing our own. Instead of traditional indistinguishability-based notions, our idea is to look at expected gain of the adversary, parametrised by a leakage function in the same way as the previous notions, a gain function which describes the goal of the adversary, distribution of databases and auxiliary information associated to the databases. In this way, we can reason exactly what is leaked under different assumptions on adversarial knowledge.

To compute expectation of our notion, we relate our notion to its idealised counterpart. It turns out that it is not always possible to bound the expectation of our notion by the idealised one. We theorise necessary conditions for this to hold. We found that our notion is closely related to g-leakage \cite{6266165} and results on g-leakage can be used to evaluate expectation in our notion. In particular, data processing inequality can be used to isolate part of the leakage, allowing one to bound expectation of our notion. However, data processing inequality in the literature is incomplete in the sense that it is one-sided. We give our results on data processing inequality to complete the theorem.
\subsection{Bounding Expectation in Our Notion}
It is hard to compute expectation of gain function in our notion in the real world, so we take an approach similar to that of SS-CKA. In this section, we will first define idealised version of our notion, and then show if the underlying scheme is secure under \textbf{S-BB-SS} then the expectation of the gain function in our notion can be bounded by that of the ideal notion.




\subsubsection{Idealised Notion}
Let $\leakm: \DB \rightarrow \{0,1\}^{*}$ be a leakage function of mechanism $\mathcal{M}$ taking input a database generated by the adversary and outputs some leakage. Let $\simer: \{0,1\}^{*} \rightarrow \EDB$ be a PPT simulator taking input some leakage and outputs an encrypted database. We define idealised version of our notion as follows.

\begin{figure}[H]
	\begin{pchstack}[center]
		\procedure{\simulator$_{\Pi, \adv, \simer}^{g}(n)$}{%
			\pcln (\db, \aux) \sample \Pi \\
			\pcln \leak \gets \leakm(\db) \\
			\pcln \edb \gets \simer(\secparam, \leak) \\
			\pcln w \gets \mathcal{A}(\secparam, \edb, \aux) \\
			\pcln \pcreturn g(w, \db)
		}
	\end{pchstack}
	\caption{Our idealised security game.}
\end{figure}




\subsubsection{\textbf{S-BB-SS} and our Notion}
One key idea we have is that we want to be compatible with previous notions as much as possible so schemes proven to be secure in those settings can be easily adapt to our notion. We observed that works using SS-CKA as their security notion usually proves security of their schemes in the black box model. In another word, security of those schemes satisfies \textbf{S-BB-SS} with adaptive queries. For simplicity of notation, we will consider passive adversary in this work, but all the results can be easily generalised to adaptive security. We will prove our desired result by using a series of lemmas.


\begin{lemma} \label{Lemma: S-BB-SS with db}
	\normalfont
	Let mechanism $\mathcal{M}$ be an SSE with security parameter $\secparam$. Suppose that it satisfies \textbf{S-BB-SS} with leakage function $\leakm$, then there exists a simulator $\simer$ such that the outputs of the following experiments are computationally indistinguishable.
	
	\begin{pchstack}[center]
		\procedure{Real$_{\mathcal{\adv}}(n)$}{%
			\pcln \key \sample \kgen(\secparam) \\
			\pcln (\db, \stateA)  \gets \adv_0(\secparam) \\
			\pcln \edb, \gets \enc(\secparam, \key, \db) \\
			\pcln s \gets \adv_1(\secparam, \edb, \stateA) \\
			\pcln \pcreturn (s,\db)
		}	
		
		\pchspace
		\procedure{$\simulator_{\mathcal{\adv,S}}(n)$}{%
			\pcln (\db, \stateA)   \gets \adv_0(\secparam) \\
			\pcln \leak \gets \leakm(\secparam, \db) \\
			\pcln \edb  \gets \mathcal{S}(\secparam, \leak) \\
			\pcln s \gets \adv_1(\secparam, \edb, \stateA) \\
			\pcln \pcreturn (s,\db)
		}
	\end{pchstack}
\end{lemma}


\begin{proof}
	Assume the contrary that the outputs of the experiments are computationally distinguishable, then there is a distinguisher $\dister$ that distinguishes the two experiments. This distinguisher can be used to break \textbf{S-BB-SS} if the first adversary's state $\stateA = \db$, which is permitted by the security notion, and the lemma is proven.
\end{proof}


\begin{lemma} \label{Lemma: S-BB-SS with Pi and db}
	\normalfont
	Let mechanism $\mathcal{M}$ be an SSE with security parameter $\secparam$. Suppose that it satisfies \textbf{S-BB-SS} with leakage function $\leakm$, then there exists a simulator $\simer$ such that the outputs of the following experiments are computationally indistinguishable.
	
	\begin{pchstack}[center]
		\procedure{Real$_{\Pi,\mathcal{\adv}}(n)$}{%
			\pcln (\db, \aux) \sample \Pi \\
			\pcln \key \sample \kgen(\secparam) \\
			\pcln \edb, \gets \enc(\secparam, \key, \db) \\
			\pcln s \gets \adv_1(\secparam, \edb, \aux) \\
			\pcln \pcreturn (s,\db)
		}	
		
		\pchspace
		\procedure{$\simulator_{\Pi,\mathcal{\adv,S}}(n)$}{%
			\pcln (\db, \aux) \sample \Pi \\
			\pcln \leak \gets \leakm(\secparam, \db) \\
			\pcln \edb  \gets \mathcal{S}(\secparam, \leak) \\
			\pcln s \gets \adv_1(\secparam, \edb, \aux) \\
			\pcln \pcreturn (s,\db)
		}
	\end{pchstack}
\end{lemma}

\begin{proof}
	The experiments above are slight modification from lemma \ref{Lemma: S-BB-SS with db} where the first adversary $\adversary_0$ is replaced by input distribution $\Pi$. As discussed before, parametrising over input distribution $\Pi$ allows us to reason about security better. In fact, security in this sense is weaker than \textbf{S-BB-SS} as $\Pi$ is essentially a restricted adversary $\adversary_0$ in lemma \ref{Lemma: S-BB-SS with db}.
\end{proof}


\begin{lemma}
	Let mechanism $\mathcal{M}$ be an SSE with security parameter $\secparam$. Suppose that it satisfies \textbf{S-BB-SS} with leakage function $\leakm$, then there exists a simulator $\simer$ such that the expectations of our real and ideal notions are close, i.e. only differed by a negligible function in the security parameter.
	
	The gain function $g$ in our notions must satisfy the following properties. Let $X$ and $Y$ be the sets of outputs of the real and ideal experiments in lemma \ref{Lemma: S-BB-SS with Pi and db} respectively. Then
	\begin{enumerate}
		\item $|g[X \cup Y]| \in \poly$,
		\item $\sup_{z} |g(z)| \in \poly$.
	\end{enumerate}
\end{lemma}

\begin{proof}
	Our experiments are nothing but the ones in \ref{Lemma: S-BB-SS with Pi and db} with gain function $g$ applied to the outputs. From the lemma above, we know that those outputs are computationally indistinguishable. So by applying corollary \ref{Corollary: computationally indistinguishable expectations}, we conclude that the expectations in our real and ideal experiments are computationally close.
\end{proof}

In the next section, we will connect our notion to information theory. In particular, we will show how to use techniques in g-leakage to compute the expectation in our notion.

